%%%%%%%%%%%%%%%%%%%%%%%%%%%%%%%%%%%%%%%%%
% Friggeri Resume/CV
% XeLaTeX Template
% Version 1.0 (5/5/13)
%
% This template has been downloaded from:
% http://www.LaTeXTemplates.com
%
% Original author:
% Adrien Friggeri (adrien@friggeri.net)
% https://github.com/afriggeri/CV
% Modfied by: Sascha Manns (Sascha.Manns@outlook.de)
% https://github.com/saigkill/latex_curriculum_vitae-dotnet
%
% License:
% CC BY-NC-SA 3.0 (http://creativecommons.org/licenses/by-nc-sa/3.0/)
%
% Important notes:
% This template needs to be compiled with XeLaTeX and the bibliography, if used,
% needs to be compiled with biber rather than bibtex.
%
%%%%%%%%%%%%%%%%%%%%%%%%%%%%%%%%%%%%%%%%%

\documentclass[a4paper]{friggeri-cv} % Add 'print' as an option into the square bracket to remove colors from this template for printing

\hypersetup{
pdftitle={Bewerbungsunterlagen}, %%
pdfauthor={Sascha Manns}, %%
pdfsubject={Bewerbungsunterlagen}, %%
pdfcreator={XeLaTEX and Biber with hyperref-package.},
pdfproducer={Sascha Manns, Mayen}, %%
pdfkeywords={Manns, Mayen, Bewerbung, IT-Support, Community, Linux, Programmierer, Dispatcher, Buchautor, Schriftsteller, Geschäftsprozess} %%
}
\usepackage{graphicx}
\usepackage{xltxtra}
\usepackage{progressbar}
\usepackage{pdfpages}

\addbibresource{./bibliography.bib} % Specify the bibliography file to include publications
\input{$HOME/AppData/Roaming/latex_curriculum_vitae/personal_data.tex}

\begin{document}

\header{\firstname}{\familyname}{Fachinformatiker Anwendungsentwicklung} % Your name and current job title/field

%----------------------------------------------------------------------------------------
%	SIDEBAR SECTION
%----------------------------------------------------------------------------------------

\begin{aside} % In the aside, each new line forces a line break
\section{Kontakt}
~
\includegraphics[width=5cm]{$HOME/AppData/Roaming/latex_curriculum_vitae/Pictures/userpic.jpg}
~
\mystreet
\mycity
Germany
~
\myphone
~
\href{mailto:Sascha.Manns@outlook.de}{Sascha.Manns@outlook.de}\includegraphics[width=0.4cm]{$HOME/AppData/Roaming/latex_curriculum_vitae/Pictures/email.png}
%\href{http://saschamanns.de}{saschamanns.de}\includegraphics[width=0.3cm]{../Pictures/aboutme.png}
Geburtsdatum: 01.10.1979
\section{Social Media}
\href{https://www.xing.com/profile/SaschaZyroslawKyrill_Manns}{Sascha\_Manns}\includegraphics[width=0.4cm]{$HOME/AppData/Roaming/latex_curriculum_vitae/Pictures/xing.png}
\href{https://www.linkedin.com/in/saigkill}{saigkill}\includegraphics[width=0.4cm]{$HOME/AppData/Roaming/latex_curriculum_vitae/Pictures/linkedin.png}
\href{https://www.facebook.com/sascha.manns}{sascha.manns}\includegraphics[width=0.4cm]{$HOME/AppData/Roaming/latex_curriculum_vitae/Pictures/facebook.png}
\href{https://twitter.com/saigkill}{saigkill}\includegraphics[width=0.4cm]{$HOME/AppData/Roaming/latex_curriculum_vitae/Pictures/twitter.png}
\href{https://wakatime.com/@saigkill}{saigkill}\includegraphics[width=0.4cm]{$HOME/AppData/Roaming/latex_curriculum_vitae/Pictures/wakatime.jpg}
\href{https://codestats.net/users/saigkill}{saigkill}\includegraphics[width=0.4cm]{$HOME/AppData/Roaming/latex_curriculum_vitae/Pictures/codestats.png}
\end{aside}

%----------------------------------------------------------------------------------------
%	WORK EXPERIENCE SECTION
%----------------------------------------------------------------------------------------

\section{Berufserfahrung}

\begin{entrylist}
%------------------------------------------------
\entry
{03/2020 -- 08/2020}
{Softwareentwickler}
{Andernach}
{Logis Softwareentwicklung GmbH\\
	Detailierte Aufgaben:
	\begin{itemize}
		\item Konzeption \& Programmierung betriebswirtschaftlicher Anwendungen und Prozesse
		\item Realisierung \& Anpassung Belegerstellung
	\end{itemize}
}
%------------------------------------------------
\entry
{03/2018 -- 12/2019}
{App \& Webentwickler}
{Mendig}
{CPS solutions GmbH \& Co KG\\
	Detailierte Aufgaben:
	\begin{itemize}
		\item Konzeption \& Programmierung mobiler Anwendungen
		\item Realisierung \& Anpassung Belegerstellung
        	\item Erstellung Dokumentation
        	\item telefonischer Support
	\end{itemize}
}
%------------------------------------------------
\entry
{10/2015 -- 02/2018}
{arbeit suchend}
{}
{}

\entry
{09/2014 -- 09/2015}
{Autor Geschäftsprozessdokumentation}
{Andernach}
{XCOM AG (jetzt Fintech) \\
	Detailierte Aufgaben:
	\begin{itemize}
		\item Autor Geschäftsprozessdokumentation (Software \& Bankwesen)
		\item Artikelerstellung und Pflege internes Wiki
		\item Produktion Präsentationen für Mandanten und interne Schulungen
		\item Modellierung von Geschäftsprozessen nach BPMN
	\end{itemize}
}
%------------------------------------------------
\entry
{02/2014 -- 05/2014}
{IT-Supporter/Dispatcher}
{Mannheim}
{Hays Temp GmbH \\
    Detailierte Aufgaben bei ITSCare Neuwied (ZBV):
    \begin{itemize}
        \item Dispatching und Teamcontroling
        \item AmSys/IDM (AOK Krankenkassen)
        \item Eröffnung, Entsperrung oder Löschung von Benutzerkonten
        \item Passwort- und Rechtevergaben
        \item SLA-konforme Bearbeitung von eingehenden Störungen (betreffend Benutzerverwaltung) im Rahmen des Incident-Managements
        \item Pflege von Logon-Prozeduren
        \item Annahme, Erfassung und Bearbeitung eingehender Benutzeranträge
        \item Benutzerkontenverwaltung im Active Directory
    \end{itemize}
}
%------------------------------------------------
\entry
{07/2013 -- 01/2014}
{Buchautor}
{Raleigh}
{Lulu Press Inc.\\
    Detailierte Aufgaben:
    \begin{itemize}
        \item Schreiben des Handbuchs
        \item Einführung des Lektorenteams in das Projekt
        \item Einführung einer Google Code-In Studentin in den Textsatz (Teilnahme als Mentor)
    \end{itemize}
}
\end{entrylist}
\begin{entrylist}
%------------------------------------------------
\entry
{08/2010 -- 06/2013}
{Community \& Support Agent}
{Homeoffice}
{open-slx GmbH \\
    Detailierte Aufgaben:
    \begin{itemize}
        \item Level 1 \& 2 Support via Telefon und Email
        \item Community Management
        \item Projektleitung deutsches openSUSE Wiki. Migration des Wikis zu neuer Struktur
        \item openSUSE Weekly News
        \item Testen von Software
        \item Webseitenbetreuung \& Social Media
        \item Dokumentation
        \item Paketverwaltung einiger RPM-Pakete
        \item openSUSE Membership Application Team
    \end{itemize}
}
%------------------------------------------------
\entry
{07/2007 -- 12/2014}
{Softwareentwickler/Teamleiter}
{Homeoffice}
{openSUSE Linux Projekt\\
    Detailierte Aufgaben:
    \begin{itemize}
        \item Leiter des openSUSE Newsletters und publikation in bis zu 12 Sprachen
        \item Erfinder eines neuen Podcastformates (Vertonte Weekly News)
        \item openSUSE Medical Project (Subprojektgründer)
        \item Beschaffung, Aktualisierung des Sourcecodes im Buildservice
        \item Compiling und Packaging neuer Binärpakete
        \item Verteilung via Buildservice
    \end{itemize}
}
%-----------------------------------------------
\entry
{07/2000 -- 06/2007}
{Ersatzdienst zum Zivildienst}
{Selters/Ts.}
{\emph{Geistlicher eines religiösen Ordens}}
%------------------------------------------------
\entry
{08/1998 -- 07/2000}
{Kaufmann im Einzelhandel}
{Mendig}
{Auto Deschner GmbH\\
    Detailierte Aufgaben:
    \begin{itemize}
        \item Einkauf \& Warenannahme \& Kontrolle
        \item Bestandsbuchungen der Waren
        \item Kundenbetreuung
        \item Rechnungserstellung
        \item Projektplanung
    \end{itemize}
}
%------------------------------------------------
\end{entrylist}

%----------------------------------------------------------------------------------------
%	EDUCATION SECTION
%----------------------------------------------------------------------------------------

\section{Aus- und Weiterbildung}

\begin{entrylist}
%------------------------------------------------
\entry
{04/2019 -- 08/2020}
{Fachinformatiker Anwendungsentwicklung}
{Fernstudium}
{\begin{itemize}
        \item Objektorientierte Programmierung mit VB: Klassen, Vererbung, UML, Windows Benutzeroberflächen, Formulare \& Grafik, Datenbanken
        \item MS Access: Grundlagen, Datenbanktheorie, VBA
    \end{itemize}
{\emph{Abschluss: Fachinformatiker Anwendungsentwicklung} (Note: 1,5)}
}
%------------------------------------------------
\newline
\entry
{11/2013 -- 12/2013}
{Web Engineering I}
{Web}
{Technische Hochschule Mittelhessen}
%-----------------------------------------------
\entry
{08/1995 -- 08/1998}
{Ausbildung zum Kaufmann um Einzelhandel}
{Mendig}
{Auto Deschner GmbH\\
Allgemeine kaufm. Tätigkeiten.
}
%-----------------------------------------------
\entry
{08/1995}
{Hauptschulabschluss}
{Mayen}
{Hauptschule Hinter-Burg}
%-----------------------------------------------
\end{entrylist}
\newpage
%----------------------------------------------------------------------------------------
% COMPUTER SKILLS
% ---------------------------------------------------------------------------------------

\section{Fremdsprachen}
Deutsch \progressbar[linecolor=blue,tickscolor=orange,emptycolor=white,filledcolor=blue]{1.0}
Englisch \progressbar[linecolor=blue,tickscolor=orange,emptycolor=white,filledcolor=blue]{0.6}

\section{IT-Kenntnisse}
\begin {itemize}
	\item Office: Microsoft365 \progressbar[linecolor=blue,tickscolor=orange,emptycolor=white,filledcolor=green]{0.8}
    	\item Cloud Computing: Microsoft Azure \progressbar[linecolor=blue,tickscolor=orange,emptycolor=white,filledcolor=green]{0.3}
	\item OS: Linux \progressbar[linecolor=blue,tickscolor=orange,emptycolor=white,filledcolor=green]{0.7}, Windows 10 \progressbar[linecolor=blue,tickscolor=orange,emptycolor=white,filledcolor=green]{0.9}
	\item CMS: Wordpress \progressbar[linecolor=blue,tickscolor=orange,emptycolor=white,filledcolor=green]{0.7}
	\item Agile: SCRUM \progressbar[linecolor=blue,tickscolor=orange,emptycolor=white,filledcolor=green]{0.7}
\end{itemize}

\section{Programmierung}
\begin{itemize}
	\item .NET Framework (C\# \& Visual Basic) \progressbar[linecolor=blue,tickscolor=orange,emptycolor=white,filledcolor=green]{0.8}
	\item .NET Core (C\#) \progressbar[linecolor=blue,tickscolor=orange,emptycolor=white,filledcolor=green]{0.4}
	\item Ruby \progressbar[linecolor=blue,tickscolor=orange,emptycolor=white,filledcolor=green]{0.4}
	\item JAVA \progressbar[linecolor=blue,tickscolor=orange,emptycolor=white,filledcolor=green]{0.2}
	\item HTML5, CSS, JavaScript \progressbar[linecolor=blue,tickscolor=orange,emptycolor=white,filledcolor=green]{0.6}
	\item SOAP \progressbar[linecolor=blue,tickscolor=orange,emptycolor=white,filledcolor=green]{0.6}
	\item Jasper Report \progressbar[linecolor=blue,tickscolor=orange,emptycolor=white,filledcolor=green]{0.6}, List \& Label \progressbar[linecolor=blue,tickscolor=orange,emptycolor=white,filledcolor=green]{0.5}
	\item TSQL/MSSQL \progressbar[linecolor=blue,tickscolor=orange,emptycolor=white,filledcolor=green]{0.9}
	\item Intersystems Caché \progressbar[linecolor=blue,tickscolor=orange,emptycolor=white,filledcolor=green]{0.5}
\end{itemize}

%----------------------------------------------------------------------------------------
%	INTERESTS SECTION
%----------------------------------------------------------------------------------------

\section{Interessen}
\begin{itemize}
  \item Neue Programmiersprachen, Frameworks und Dienste lernen
  \item Bringe gerne anderen etwas neues bei
  \item Reise gerne in andere Länder um neue Impulse zu bekommen
  \item Mit der Familie spazieren gehen
  \item Mitgliedschaft in der Fachgruppe "Wirtschaftsinformatik" der Gesellschaft für Informatik
\end{itemize}

%----------------------------------------------------------------------------------------
%	PUBLICATIONS SECTION
%----------------------------------------------------------------------------------------
%\section{Publikationen}

%\printbibsection{article}{Artikel} % Print all articles from the bibliography

%\printbibsection{book}{Bücher} % Print all books from the bibliography

%\printbibsection{misc}{other publications} % Print all miscellaneous entries from the bibliography

%\printbibsection{report}{research reports} % Print all research reports from the bibliography

%----------------------------------------------------------------------------------------

\begin{center}
\includegraphics[scale=0.7]{$HOME/AppData/Roaming/latex_curriculum_vitae/Pictures/signatur.png} \\
\begin{tabular}{@{}l@{}}
\\ $\frac{}{\strut\textnormal{Sascha Manns, Mayen den \today}}$
\end{tabular}
\end{center}

%\section{Anhang}
%---------------------------------------------------------------------------
\end{document}
